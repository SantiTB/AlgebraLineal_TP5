\documentclass{article}
\usepackage[margin=0.5in]{geometry}
\usepackage{amsmath}
\usepackage{graphicx}
\usepackage{amssymb}
\usepackage{multicol}
\usepackage{xcolor}
\usepackage{amsthm}
\usepackage{mdframed}

\newenvironment{tightcenter}{%
    \setlength\topsep{0pt}
    \setlength\parskip{0pt}
    \begin{center}
}{%  
    \end{center}
}
\newcommand{\R}{\mathbb{R}}
\newcommand{\Z}{\mathbb{Z}}
\newcommand{\C}{\mathbb{C}}
\newcommand{\K}{\mathbb{K}}
\newcommand{\PP}{\mathbb{P}}
\newcommand{\E}{\mathcal{E}}
\title{Trabajo Práctico 5 - Formas canónicas elementales I}
\author{Santiago}
\date{}
\begin{document}
    \maketitle
    \global\mdfdefinestyle{s}{%
            linecolor=orange,linewidth=0.5pt,%
            leftmargin=0cm,rightmargin=1cm
        }
    \begin{enumerate}
        \item Sea \[A=\begin{pmatrix}
        2&-1&0\\-1&2&-1\\0&-1&2
    \end{pmatrix}\]Probar que $2,2+\sqrt{2}$ y $2-\sqrt{2}$ son autovalores de $A$ y hallar los autovectores correspondientes.
    \begin{mdframed}[style=s]
        
    \end{mdframed}
        \item Sea $A\in\K^{n\times n}$ una matriz inversible ¿Puede ser $\lambda=0$ un autovalor de $A$? Probar que si $\lambda$ es un autovalor de $A$, entonces $\lambda^{-1}$ es autovalor de $A^{-1}$ y además los autoespacios asociados a $\lambda$ y $\lambda^{-1}$ pertenecientes a $A$ y $A^{-1}$ respectivamente, coinciden.
    \begin{mdframed}[style=s]
        Si $\lambda=0$ es autovalor $Av=0\to A^{-1}Av=0\to v=0$. Esto contradice la definición de autovalor, ya que $v\neq 0$. Por lo tanto, $\lambda$ no puede ser $0$.\\
        Si $\lambda$ es autovalor de $A\to Av=\lambda v\to A^{-1}Av=A^{-1}\lambda v\to Iv=\lambda A^{-1}v\to \frac{1}{\lambda}Iv=A^{-1}v\to \lambda^{-1}$ es autovalor de $A^{-1}$.\\
        Los autoespacios asociados a $\lambda$ y $\lambda^{-1}$ son\[E(\lambda)=N(A-\lambda I)\qquad E(\lambda^{-1})=N(A^{-1}-\lambda^{-1}I)\]
        Los vectores $v$ del autoespacio asociado a $\lambda$ cumplen\[(A-\lambda I)v=0\to Av=\lambda Iv\to Av=\lambda v\]
        Es decir, los vectores del autoespacio, son los autovectores. Por otra parte, los vectores $w$ del autoespacio asociado a $\lambda^{-1}$ cumplen\[(A^{-1}-\lambda^{-1} I)w=0\to A^{-1}w=\lambda^{-1} Iw\to A^{-1}w=\lambda^{-1} w\to w=\lambda^{-1}Aw\to Aw=\lambda w\]
        que es exactamente la misma condición de los elementos del autoespacio asociado a $\lambda$, con lo cual estos coinciden.
    \end{mdframed}
        \item Probar que si $A\in\K^{n\times n}$ es una matriz triangular, entonces los autovalores de $A$ son los elementos de la diagonal.
    \begin{mdframed}[style=s]
        
    \end{mdframed}
        \item Sea $A\in\K^{n\times n}$ ¿Puede tener $A$ más de $n$ autovectores linealmente independientes?
    \begin{mdframed}[style=s]
        Los autovalores de $A$ son las raíces del polinomio característico, el mismo es un polinomio de grado $n$, entonces voy a tener a lo sumo $n$ soluciones. Por lo tanto, no es posible que hayan más de $n$.\\
        También se puede pensar a $A$ como la representación matricial de alguna transformación lineal 
        \[T:V\to V\quad dim(V)=n\]
        por el Corolario del \textbf{Lema 4.9} $T$ tiene a lo sumo $n$ autovalores distintos.
    \end{mdframed}
        \item \begin{enumerate}
        \item Construir una matriz $A\in\R^{2\times2}$ que tenga un sólo autovalor.
            \begin{mdframed}[style=s]
                
            \end{mdframed}
        \item Construir una matriz $A\in\R^{2\times2}$ que tenga un sólo autovalor con un autoespacio asociado de dimensión 1.
            \begin{mdframed}[style=s]
                
            \end{mdframed}
        \item Contruir una matriz $A\in\R^{2\times2}$ que no tenga autovalores. ¿Puede hacer lo mismo para una matriz de $\C^{2\times2}$?
            \begin{mdframed}[style=s]
                
            \end{mdframed}
    \end{enumerate}
        \item Considerar las transformaciones lineales $R_\frac{\pi}{2},S_Y,H_2$ y $P_X$ del ejercicio 11 de la práctica 2. Hallar los autovalores, autovectores y autoespacios asociados. ¿Es alguna de ellas diagonalizable?
    \begin{mdframed}[style=s]
        
    \end{mdframed}
        \item Sea $V$ un $\K$-EV de dimensión finita y sea $T\in L(V)$. Probar que, si $\lambda$ y $\mu$ son dos autovalores de $T$ diferentes, entonces $N(T-\lambda I)\cap N(T-\mu I)=\{\vec{0}\}$.
    \begin{mdframed}[style=s]
        Por el \textbf{Lema 4.9} los autovectores asociados a los distintos autovalores son li, entonces $\overline{\{v_\lambda\}}\cap\overline{\{v_\mu\}}=\{\vec{0}\}\to E(\lambda)\cap E(\mu)=\{\vec{0}\}\to N(T-\lambda I)\cap N(T-\mu I)=\{\vec{0}\}$
    \end{mdframed}
        \item Sea $V$ un $\K$-EV de dimensión finita y sea $T\in L(V)$. Supongamos que $\lambda$ es un autovalor de $T$ y que $v\in V$ es un autovector asociado a $\lambda$. Probar que si $p\in\K[x]$, entonces $p(T)v=p(\lambda)v$.
    \begin{mdframed}[style=s]
        
    \end{mdframed}
        \item Sea \[A=\begin{pmatrix}
        0&-2&1\\1&0&0\\0&0&1
    \end{pmatrix}\]
    \begin{enumerate}
        \item Hallar el polinomio minimal de $A$ considerando los coeficientes en $\C$, en $\R$ y en $\Z_3$.\pagebreak
            \begin{mdframed}[style=s]
                Como el minimal tiene las mismas raíces que el polinomio característico, empiezo calculando éste último
                \[p_A(\lambda)=det(\lambda I-A)=\begin{vmatrix}
                    \lambda&2&-1\\-1&\lambda&0\\0&0&\lambda-1
                \end{vmatrix}=(\lambda-1)\begin{vmatrix}
                    \lambda&2\\-1&\lambda
                \end{vmatrix}=(\lambda-1)(\lambda^2+2)\]
                Para $\C$: $p_A(\lambda)=(\lambda-1)(\lambda-2i)(\lambda+2i)\to m_A(\lambda)=p_A(\lambda)$.\\
                En el caso de $\R$, recordando que el minimal divide al característico y comparte raíces, concluimos que hay 2 posibilidades \[m_A(\lambda)=p_A(\lambda),\quad m_A(\lambda)=(\lambda-1)\]
                Sin embargo, $(\lambda-1)$ no anula a $A$, con lo cual $m_A(\lambda)=p_A(\lambda)$.
            \end{mdframed}
        \item Decir en cada caso si $A$ es diagonalizable.
            \begin{mdframed}[style=s]
                Por el \textbf{Teorema 4.41} en $\C$ es diagonalizable, mientras que en $\R$ no.
            \end{mdframed}
    \end{enumerate}
        \item Para cada una de las siguientes matrices hallar sus autovalores y autoespacios asociados. Decir si son diagonalizables y en caso de serlo hallar la matriz diagonal y el cambio de base correspondiente.
    \[A=\begin{pmatrix}
        1&1&1\\0&1&2\\0&0&1
    \end{pmatrix},\quad B=\begin{pmatrix}
        1&-3&3\\3&-5&3\\6&-6&4
    \end{pmatrix},\quad C=\begin{pmatrix}
        -3&0&0\\0&5&-1\\0&6&-2
    \end{pmatrix}.\]
    \begin{mdframed}[style=s]
        \begin{itemize}
            \item $A:$\\
                Como la matriz es triangular, los autovalores son los elementos de la diagonal principal. En este caso el único autovalor es $\lambda=1$. Para hallar el autoespacio asociado, tengo que resolver\[\left(\begin{pmatrix}
                    1&1&1\\0&1&2\\0&0&1
                \end{pmatrix}-\lambda\begin{pmatrix}
                    1&0&0\\0&1&0\\0&0&1
                \end{pmatrix}\right)\begin{pmatrix}
                    x\\y\\z
                \end{pmatrix}=\begin{pmatrix}
                    0\\0\\0
                \end{pmatrix}\to y=z=0\]
                Por lo tanto, el autoespacio asociado será: \[E(1)=\overline{\{(1,0,0)\}}\]
                No hay base de autovectores, entonces no es diagonalizable.
            \item $B:$\\
                El polinomio característico es\[p_B(x)=(x+2)^2(x-4)\]
                Por lo tanto, los autovalores son\[\lambda_1=-2\qquad\lambda_2=4\]
                Por otra parte, \[E(-2)=\overline{\{(-1,0,1),(1,1,0)\}}\qquad E(4)=\overline{\{(1,1,2)\}}\]
                Como hay una base de autovectores, entonces $B$ es diagonalizable.
                \[B=\begin{pmatrix}
                    -1&1&1\\
                    0 &1&1\\
                    1 &0&2\\
                \end{pmatrix}\begin{pmatrix}
                    -2&0&0\\0&-2&0\\0&0&4
                \end{pmatrix}\begin{pmatrix}
                    -1&1&1\\
                    0 &1&1\\
                    1 &0&2\\
                \end{pmatrix}^{-1}\]
            \item $C:$
                El polinomio característico es:\[p_C(x)=(\lambda+3)(\lambda-4)(\lambda+1)\]
                Los autovalores son \[\lambda_1=-3\qquad \lambda_2=4\qquad \lambda_3=-1\]
                Los autoespacios son \[E(-3)=\overline{\{(1,0,0)\}}\qquad E(4)=\overline{\{(0,1,1)\}}\qquad E(-1)=\overline{\{(0,1,6)\}}\]
                La diagonalización es igual que en el caso anterior.
        \end{itemize}
    \end{mdframed}
        \item Considerar las matrices \[A=\begin{pmatrix}
        1&1&0\\0&2&0\\0&0&1
    \end{pmatrix},\quad B=\begin{pmatrix}
        2&0&0\\0&2&2\\0&0&1
    \end{pmatrix}.\]
    Probar que $A$ y $B$ tienen polinomios característicos diferentes, pero sus polinomios minimales coinciden.
    \begin{mdframed}[style=s]
        
    \end{mdframed}
        \item Sea $T\in L(\R_2[x])$ dado por $T(a_0+a_1x+a_2x^2)=(a_0+a_1)-(2a_1+3a_2)x$.
    \begin{enumerate}
        \item Hallar la representación matricial de $T$ con respecto a la base usual de $R_2[x]$.
            \begin{mdframed}[style=s]
                
            \end{mdframed}
        \item Hallar el polinomio característico y los autovalores de $T$.
            \begin{mdframed}[style=s]
                
            \end{mdframed}
        \item ¿Es $T$ diagonalizable?
            \begin{mdframed}[style=s]
                
            \end{mdframed}
    \end{enumerate}
        \item Sea $T\in L(\R^3)$ dado por $T(x,y,z)=(x,x+y,z)$.
    \begin{enumerate}
        \item Hallar el polinomio característico y el minimal de $T$.
            \begin{mdframed}[style=s]
                
            \end{mdframed}
        \item Calcular los autovalores y una base para cada autoespacio asociado.
            \begin{mdframed}[style=s]
                
            \end{mdframed}
        \item Decir si $T$ es diagonalizable, justificando de dos maneras diferentes.
            \begin{mdframed}[style=s]
                
            \end{mdframed}
    \end{enumerate}
        \item \begin{enumerate}
        \item Sea $T\in L(\R^3)$ tal que:
            \begin{itemize}
                \item Sus autovalores son 1 y -1.
                \item $\{(0,1,-1)\}$ es una base de $N(T+I)$ y $\{(0,1,1);(1,0,0)\}$ es una base de $N(T-I)$.
            \end{itemize}
            ¿Se puede decir si $T$ es diagonalizable? Hallar el polinomio característico de $T$.
            \begin{mdframed}[style=s]
                
            \end{mdframed}
        \item Sea $T\in L(\R^4)$ tal que:
            \begin{itemize}
                \item Sus autovalores son 1 y -1.
                \item $\{(0,-1,0,0)\}$ es una base de $N(T+I)$ y $\{(0,0,1,1);(1,0,0,0)\}$ es una base de $N(T-I)$.
            \end{itemize}
            ¿Se puede decir si $T$ es diagonalizable?
            \begin{mdframed}[style=s]
                
            \end{mdframed}
    \end{enumerate}
        \item ¿Cuáles son los posibles autovalores de una matriz $A$ si se sabe que $A=A^2$?
    \begin{mdframed}[style=s]
        
    \end{mdframed}
        \item Decir para qué valores de $a$ y $b$ la siguiente matriz es diagonalizable.\[A=\begin{pmatrix}
        a&b&0\\0&-1&0\\0&0&1
    \end{pmatrix}.\]
    \begin{mdframed}[style=s]
        
    \end{mdframed}
        \item Sea $A$ una matriz cuadrada tal que $A\neq I$ y $A^3-A^2+A=I$. ¿Es $A$ diagonalizable sobre $\C$?¿Y sobre $\R$?
    \begin{mdframed}[style=s]
        
    \end{mdframed}
        \item Probar que si $A\in\R^{2\times2}$ es simétrica, entonces es semejante a una matriz diagonal.
    \begin{mdframed}[style=s]
        
    \end{mdframed}
    \end{enumerate}
\end{document}