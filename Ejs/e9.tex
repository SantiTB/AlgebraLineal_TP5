\item Sea \[A=\begin{pmatrix}
        0&-2&1\\1&0&0\\0&0&1
    \end{pmatrix}\]
    \begin{enumerate}
        \item Hallar el polinomio minimal de $A$ considerando los coeficientes en $\C$, en $\R$ y en $\Z_3$.\pagebreak
            \begin{mdframed}[style=s]
                Como el minimal tiene las mismas raíces que el polinomio característico, empiezo calculando éste último
                \[p_A(\lambda)=det(\lambda I-A)=\begin{vmatrix}
                    \lambda&2&-1\\-1&\lambda&0\\0&0&\lambda-1
                \end{vmatrix}=(\lambda-1)\begin{vmatrix}
                    \lambda&2\\-1&\lambda
                \end{vmatrix}=(\lambda-1)(\lambda^2+2)\]
                Para $\C$: $p_A(\lambda)=(\lambda-1)(\lambda-2i)(\lambda+2i)\to m_A(\lambda)=p_A(\lambda)$.\\
                En el caso de $\R$, recordando que el minimal divide al característico y comparte raíces, concluimos que hay 2 posibilidades \[m_A(\lambda)=p_A(\lambda),\quad m_A(\lambda)=(\lambda-1)\]
                Sin embargo, $(\lambda-1)$ no anula a $A$, con lo cual $m_A(\lambda)=p_A(\lambda)$.
            \end{mdframed}
        \item Decir en cada caso si $A$ es diagonalizable.
            \begin{mdframed}[style=s]
                Por el \textbf{Teorema 4.41} en $\C$ es diagonalizable, mientras que en $\R$ no.
            \end{mdframed}
    \end{enumerate}