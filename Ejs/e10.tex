\item Para cada una de las siguientes matrices hallar sus autovalores y autoespacios asociados. Decir si son diagonalizables y en caso de serlo hallar la matriz diagonal y el cambio de base correspondiente.
    \[A=\begin{pmatrix}
        1&1&1\\0&1&2\\0&0&1
    \end{pmatrix},\quad B=\begin{pmatrix}
        1&-3&3\\3&-5&3\\6&-6&4
    \end{pmatrix},\quad C=\begin{pmatrix}
        -3&0&0\\0&5&-1\\0&6&-2
    \end{pmatrix}.\]
    \begin{mdframed}[style=s]
        \begin{itemize}
            \item $A:$\\
                Como la matriz es triangular, los autovalores son los elementos de la diagonal principal. En este caso el único autovalor es $\lambda=1$. Para hallar el autoespacio asociado, tengo que resolver\[\left(\begin{pmatrix}
                    1&1&1\\0&1&2\\0&0&1
                \end{pmatrix}-\lambda\begin{pmatrix}
                    1&0&0\\0&1&0\\0&0&1
                \end{pmatrix}\right)\begin{pmatrix}
                    x\\y\\z
                \end{pmatrix}=\begin{pmatrix}
                    0\\0\\0
                \end{pmatrix}\to y=z=0\]
                Por lo tanto, el autoespacio asociado será: \[E(1)=\overline{\{(1,0,0)\}}\]
                No hay base de autovectores, entonces no es diagonalizable.
            \item $B:$\\
                El polinomio característico es\[p_B(x)=(x+2)^2(x-4)\]
                Por lo tanto, los autovalores son\[\lambda_1=-2\qquad\lambda_2=4\]
                Por otra parte, \[E(-2)=\overline{\{(-1,0,1),(1,1,0)\}}\qquad E(4)=\overline{\{(1,1,2)\}}\]
                Como hay una base de autovectores, entonces $B$ es diagonalizable.
                \[B=\begin{pmatrix}
                    -1&1&1\\
                    0 &1&1\\
                    1 &0&2\\
                \end{pmatrix}\begin{pmatrix}
                    -2&0&0\\0&-2&0\\0&0&4
                \end{pmatrix}\begin{pmatrix}
                    -1&1&1\\
                    0 &1&1\\
                    1 &0&2\\
                \end{pmatrix}^{-1}\]
            \item $C:$
                El polinomio característico es:\[p_C(x)=(\lambda+3)(\lambda-4)(\lambda+1)\]
                Los autovalores son \[\lambda_1=-3\qquad \lambda_2=4\qquad \lambda_3=-1\]
                Los autoespacios son \[E(-3)=\overline{\{(1,0,0)\}}\qquad E(4)=\overline{\{(0,1,1)\}}\qquad E(-1)=\overline{\{(0,1,6)\}}\]
                La diagonalización es igual que en el caso anterior.
        \end{itemize}
    \end{mdframed}