\item Probar que si $A\in\K^{n\times n}$ es una matriz triangular, entonces los autovalores de $A$ son los elementos de la diagonal.
    \begin{mdframed}[style=s]
        Hay 2 tipos de matrices triangulares: superior e inferior. Supongamos una superior:
        \[A=\begin{pmatrix}
            a_{11}&a_{12}&a_{13}&\dots &a_{1n}\\
            0     &a_{22}&a_{23}&\dots &a_{2n}\\
            0     &0     &a_{33}&\dots &a_{3n}\\
            \vdots&\vdots&\vdots&\ddots&\vdots\\
            0     &0     &0     &\dots &a_{nn}
        \end{pmatrix}\]
        Para encontrar los autovalores, tengo que encontrar las raíces del polinomio característico:
        \[p(\lambda)=det(\lambda I-A)=\begin{vmatrix}
            \lambda-a_{11}&a_{12}        &a_{13}        &\dots &a_{1n}\\
            0             &\lambda-a_{22}&a_{23}        &\dots &a_{2n}\\
            0             &0             &\lambda-a_{33}&\dots &a_{3n}\\
            \vdots        &\vdots        &\vdots        &\ddots&\vdots\\
            0             &0             &0             &\dots &\lambda-a_{nn}
        \end{vmatrix}\]
        Al desarrolar el determinante por la primera columna:
        \[p(\lambda)=(\lambda-a_{11})\begin{vmatrix}
            \lambda-a_{22}&a_{23}        &\dots &a_{2n}\\
            0             &\lambda-a_{33}&\dots &a_{3n}\\
            \vdots        &\vdots        &\ddots&\vdots\\
            0             &0             &\dots &\lambda-a_{nn}
        \end{vmatrix}\]
        Al repetir este proceso $n$ veces, termino con:
        \[p(\lambda)=(\lambda-a_{11})(\lambda-a_{22})(\lambda-a_{33})\dots(\lambda-a_{nn})\]
        Es decir, los autovalores de $A$ son los elementos de la diagonal principal.\\
        Para las matrices triangular inferiores, se puede llegar a la misma conclusión con un análisis similar.
    \end{mdframed}