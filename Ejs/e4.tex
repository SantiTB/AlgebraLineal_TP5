\item Sea $A\in\K^{n\times n}$ ¿Puede tener $A$ más de $n$ autovectores linealmente independientes?
    \begin{mdframed}[style=s]
        Los autovalores de $A$ son las raíces del polinomio característico, el mismo es un polinomio de grado $n$, entonces voy a tener a lo sumo $n$ soluciones. Por lo tanto, no es posible que hayan más de $n$.\\
        También se puede pensar a $A$ como la representación matricial de alguna transformación lineal 
        \[T:V\to V\quad dim(V)=n\]
        por el Corolario del \textbf{Lema 4.9} $T$ tiene a lo sumo $n$ autovalores distintos.
    \end{mdframed}